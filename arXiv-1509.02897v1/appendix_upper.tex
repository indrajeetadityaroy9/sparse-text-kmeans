\section{Proof of Theorem~\ref{rho_upper}}
\label{app_upper}

In this section we complete the proof of Theorem~\ref{rho_upper},
following the outline from Section~\ref{sec:crossPoly}. Our starting
point is the collision probability bound from Eqn.~\eqref{prob_exp}.

For $u, v \in \Rbb$ with $u \geq 0$ and $\alpha u + \beta v \geq 0$ define,
$$
\sigma(u, v) = \underset{X_2, Y_2 \sim N(0, 1)}{\mathrm{Pr}}[|X_2| \leq u \mbox{ and } |\alpha X_2 + \beta Y_2| \leq \alpha u + \beta v].
$$
Then, the right-hand side of~(\ref{prob_exp}) is equal to
$$
2d \cdot \underset{X_1, Y_1 \sim N(0, 1)}{\mathrm{E}}[\sigma(X_1, Y_1)^{d-1}].
$$
Let us define
$$
\Delta(u, v) = \min\{u, \alpha u + \beta v\}.
$$

\begin{lemma}
  \label{est1}
  For every $0 < \eps < 1/3$,
  $$
  1 - e^{-\Delta(u, v)^2 / 2} \leq \sigma(u, v) \leq 1 - \Omega\left(\eps^{1/2} \cdot e^{-(1 +\eps)\Delta(u, v)^2 / 2 }\right).
  $$
\end{lemma}
\begin{proof}
  This is a combination of Lemma~\ref{measure_convex} together with the following obvious observation: the distance from the origin to the
  set $\{(x, y) : |x| \geq u \mbox{ or } |\alpha x + \beta y| \geq \alpha u + \beta v\}$ is equal to $\Delta(u, v)$ (see Figure~\ref{gaussian_set}).
\end{proof}

\begin{lemma}
  \label{est2}
  For every $t \geq 0$ and $0 < \eps < 1/3$,
  $$
  \Omega_{\tau}\left(\eps \cdot e^{-(1 + \eps) \cdot \frac{4}{4 - \tau^2} \cdot \frac{t^2}{2}}\right) \leq \underset{X_1, Y_1 \sim N(0, 1)}{\mathrm{Pr}}[\Delta(X_1, Y_1) \geq t] \leq e^{-\frac{4}{4 - \tau^2} \cdot \frac{t^2}{2}}.
  $$
\end{lemma}
\begin{proof}
  Similar to the previous lemma, this is a consequence of Lemma~\ref{measure_wedge} together with the fact that the squared distance from the origin to the set
  $\{(x, y) \colon x \geq t \mbox{ and } \alpha x + \beta y \geq t\}$ is equal to $\frac{4}{4 - \tau^2} \cdot t^2$.
\end{proof}

\subsection{Idealized proof}

Let us expand Eqn.~\eqref{prob_exp} further, assuming that the
``idealized'' versions of Lemma~\ref{est1} and Lemma~\ref{est2}
hold. Namely, we assume that
\begin{equation}
\label{ideal_est1}
\sigma(u, v) = 1 - e^{-\Delta(u, v)^2 / 2};
\end{equation}
and
\begin{equation}
\label{ideal_est2}
\underset{X_1, Y_1 \sim N(0, 1)}{\mathrm{Pr}}[\Delta(X_1, Y_1) \geq t] = e^{-\frac{4}{4 - \tau^2} \cdot \frac{t^2}{2}}.
\end{equation}

In the next section we redo the computations using the precise bounds for
$\sigma(u, v)$ and ${\mathrm{Pr}}[\Delta(X_1, Y_1) \geq t]$.

Expanding Eqn.~\eqref{prob_exp}, we have
\begin{align}
  \underset{X_1, Y_1 \sim N(0, 1)}{\mathrm{E}}[\sigma(X_1, Y_1)^{d-1}] \nonumber & =
  \int_0^1 \underset{X_1, Y_1 \sim N(0, 1)}{\mathrm{Pr}}[\sigma(X_1, Y_1) \geq t^{\frac{1}{d-1}}] \, dt \\
  \nonumber & = 
  \int_0^1 \underset{X_1, Y_1 \sim N(0, 1)}{\mathrm{Pr}}[e^{-\Delta(X_1, Y_1)^2 / 2} \leq 1 - t^{\frac{1}{d-1}}] \, dt \\
  \nonumber & = 
  \int_0^1 (1 - t^{\frac{1}{d - 1}})^{\frac{4}{4 - \tau^2}} \, dt \\
  \nonumber & =
  (d - 1) \cdot \int_0^1 (1 - u)^{\frac{4}{4 - \tau^2}} u^{d-2} \, dt \\
  \nonumber & = (d - 1) \cdot B \left(\frac{8 - \tau^2}{4 - \tau^2}; d - 1\right) \\
  & = \Theta_{\tau}(1) \cdot d^{- \frac{4}{4 - \tau^2}}, \label{ideal_derivation}
\end{align}
where:
\begin{itemize}
\item the first step is a standard expansion of an expectation;
\item the second step is due to~(\ref{ideal_est1});
\item the third step is due to~(\ref{ideal_est2});
\item the fourth step is a change of variables;
\item the fifth step is a definition of the Beta function;
\item the sixth step is due to the Stirling approximation.
\end{itemize}
Overall, substituting~(\ref{ideal_derivation}) into~(\ref{prob_exp}), we get:
$$
\ln \frac{1}{\underset{h \sim \mathcal{H}}{\mathrm{Pr}}[h(p) = h(q)]} = \frac{\tau^2}{4 - \tau^2} \cdot \ln d \pm O_{\tau}(1).
$$

\subsection{The real proof}

We now perform the exact calculations, using the bounds (involving
$\eps$) from Lemma~\ref{est1} and Lemma~\ref{est2}. We set $\eps = 1 /
d$ and obtain the following asymptotic statements:
$$
\sigma(u, v) = 1 - d^{\pm O(1)} \cdot e^{-(1 \pm d^{-\Omega(1)}) \cdot \Delta(u, v)^2 / 2};
$$
and
$$
\underset{X, Y \sim N(0, 1)}{\mathrm{Pr}}[\Delta(X, Y) \geq t] = d^{\pm O(1)} \cdot e^{-(1 \pm d^{-\Omega(1)}) \cdot \frac{4}{4 - \tau^2} \cdot \frac{t^2}{2}}.
$$
Then, we can repeat the ``idealized'' proof (see Eqn.~\eqref{ideal_derivation}) verbatim with the new estimates and obtain the final form of Theorem~\ref{rho_upper}:
$$
\ln \frac{1}{\underset{h \sim \mathcal{H}}{\mathrm{Pr}}[h(p) = h(q)]} = \frac{\tau^2}{4 - \tau^2} \cdot \ln d \pm O_{\tau}(\ln \ln d).
$$
Note the difference in the low order term between idealized and the real version. As we argue in Section~\ref{sec_lower}, the latter
$O_{\tau}(\ln \ln d)$ is, in fact, tight.

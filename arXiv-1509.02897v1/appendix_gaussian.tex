\section{Gaussian measure of a planar set}
\label{app_gaussian}

In this Section we formalize the intuition that the standard Gaussian measure of a closed subset $A \subseteq \Rbb^2$
behaves like $e^{-\Delta_A^2 / 2}$, where $\Delta_A$ is the distance from the origin to $A$, unless $A$ is quite special.

For a closed subset $A \subseteq \Rbb^2$ and $r > 0$ denote $0 \leq \mu_A(r) \leq 1$ the normalized measure of the intersection
$A \cap r S^1$ ($A$ with the circle centered in the origin and of radius $r$):
$$
\mu_A(r) := \frac{\mu(A \cap r S^1)}{2 \pi r};
$$
here $\mu$ is the standard one-dimensional Lebesgue measure (see Figure~\ref{fig2a}). Denote $\Delta_A := \inf \{r > 0 : \mu_A(r) > 0\}$
the (essential) distance from the origin to $A$. Let $\mathcal{G}(A)$ be the standard Gaussian measure of $A$.

\begin{lemma}
  \label{measure_typical}
  Suppose that $A \subseteq \Rbb^2$ is a closed set such that $\mu_A(r)$ is non-decreasing. Then,
  $$
  \sup_{r > 0} \Bigl(\mu_A(r) \cdot e^{-r^2 / 2}\Bigr) \leq \mathcal{G}(A) \leq e^{-\Delta_A^2 / 2}.
  $$
\end{lemma}
\begin{proof}
  For the upper bound, we note that
  $$
  \mathcal{G}(A) = \int_{0}^{\infty} \mu_A(r) \cdot re^{-r^2/2} \, dr
  \leq \int_{\Delta_A}^{\infty} re^{-r^2 / 2} \, dr = e^{-\Delta_A^2 / 2}.
  $$
  For the lower bound, we similarly have, for every $r^* > 0$,
  $$
  \mathcal{G}(A) = \int_{0}^{\infty} \mu_A(r) \cdot r e^{-r^2 / 2} \, dr \geq
  \mu_A(r^*) \cdot \int_{r^*}^{\infty} re^{-r^2 / 2} \, dr = \mu_A(r^*) e^{-(r^*)^2 / 2},
  $$
  where we use that $\mu_A(r^*)$ is non-decreasing.
\end{proof}

Now we derive two corollaries of Lemma~\ref{measure_typical}.

\begin{lemma}
  \label{measure_convex}
  Let $K \subseteq \Rbb^2$ be the complement of an open convex subset of the plane that is symmetric around the origin.
  Then, for every $0 < \eps < 1/3$,
  $$
  \Omega\Bigl(\eps^{1/2} \cdot e^{-(1 + \eps) \cdot \Delta_K^2 / 2}\Bigr) \leq \mathcal{G}(K) \leq e^{-\Delta_K^2 / 2}.
  $$
\end{lemma}
\begin{proof}
  This follows from Lemma~\ref{measure_typical}: indeed, due to the convexity of the complement of $K$, $\mu_K(r)$ is non-decreasing.
  It is easy to check that
  $$
  \mu_K\Bigl((1 + \eps) \Delta_K\Bigr) = \Omega\Bigl(\eps^{1/2}\Bigr),
  $$
  again, due to the convexity (see Figure~\ref{fig2b}).
  Thus, the required bounds follow.
\end{proof}

\begin{lemma}
  \label{measure_wedge}
  Let $K \subseteq \Rbb^2$ be an intersection of two closed half-planes such that:
  \begin{itemize}
  \item $K$ does not contain a line;
  \item the ``corner'' of $K$ is the closest point of $K$ to the origin;
  \item the angle between half-planes equals to $0 < \alpha < \pi$.
  \end{itemize}
  Then, for every $0 < \eps < 1/2$,
  $$
  \Omega_{\alpha}\Bigl(\eps \cdot e^{-(1 + \eps) \cdot \Delta_K^2}\Bigr)\leq \mathcal{G}(K) \leq e^{-\Delta_K^2 / 2}.
  $$
\end{lemma}

\begin{proof}
  This, again, follows from Lemma~\ref{measure_typical}. The second condition implies that $\mu_K(r)$ is non-decreasing,
  and an easy computation shows that
  $$
  \mu_K((1 + \eps) \Delta_K) \geq \Omega_{\alpha}(\eps)
  $$
  (see Figure~\ref{fig2c}).
\end{proof}

\begin{figure}[h]
  \caption{}
  \centering
  \begin{subfigure}{0.45\textwidth}
    \includegraphics[width=\textwidth]{pictures/gaussian1.pdf}
    \caption{Defintion of $\mu_A(r)$}
    \label{fig2a}
  \end{subfigure}
  \hfill
  \begin{subfigure}{0.45\textwidth}
    \includegraphics[width=\textwidth]{pictures/gaussian2.pdf}
    \caption{For Lemma~\ref{measure_convex}}
    \label{fig2b}
  \end{subfigure}
  
  \begin{subfigure}{0.45\textwidth}
    \includegraphics[width=\textwidth]{pictures/gaussian3.pdf}
    \caption{For Lemma~\ref{measure_wedge}}
    \label{fig2c}
  \end{subfigure}
\end{figure}
